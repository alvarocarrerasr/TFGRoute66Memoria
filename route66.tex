\documentclass[twoside]{report}
\usepackage[utf8]{inputenc}
\usepackage[spanish]{babel}
\usepackage{amsmath}
\usepackage{amsfonts}
\usepackage{amssymb}
\usepackage{lipsum}
\usepackage{url}
\usepackage{cite}
\usepackage{layout}
\renewcommand{\baselinestretch}{1.2}
\setlength{\parindent}{0.5cm}
\usepackage{helvet}
\renewcommand{\familydefault}{\sfdefault}
\usepackage{graphicx}
\renewcommand{\familydefault}{\sfdefault}

\usepackage[margin=1.5cm]{geometry}

\usepackage{fancyhdr}
\fancyhf{}
\renewcommand\headrulewidth{0pt}
\fancyfoot[RO, LE]{\thepage}

\fancypagestyle{plain}{%
  \fancyhf{}
  \fancyfoot[RO, LE]{\thepage}
}


\pagestyle{fancy}

\author{Álvaro Carreras Regorigo}
\title{Route 66}
\begin{document}
\rule{\oddsidemargin}{2cm}
\begin{titlepage}
\begin{center}
\includegraphics[scale=0.5]{images/logoUVa}\\\vspace{1cm}
\begin{LARGE}\textbf{Universidad de Valladolid}\end{LARGE}\\
\vspace{2cm}
\begin{Huge}Escuela de Ingeniería Informática\end{Huge} \\\vspace{0.5cm}
\begin{large}\textbf{TRABAJO FIN DE GRADO}\end{large}\\ \vspace{2.5cm}
\begin{Large}Grado en Ingeniería Informática \\ (Mención en Ingeniería de Software)\end{Large}\\ \vspace{4cm}
\begin{Huge}\textbf{Route66App}\end{Huge}
\end{center}\vspace{6cm}
\begin{flushright}
\begin{large}Autor: \\\textbf{D. Álvaro Carreras Regorigo}\\
Tutora:\\
Dña. Margarita Gonzalo Tasis\end{large}
\end{flushright}
\end{titlepage}
\clearpage

\tableofcontents

\chapter{Parte I: Introducción y contexto.}
\section{Introducción y objetivos.}
\subsection{Introducción.}
\subsection{Motivación.}
\subsection{Objetivos.}
\section{Ludificación}
\subsection{La ludificación}
La ludificación o gamificación (del inglés, gamification) consiste, según IEBSchool en “el uso de mecánicas de juego en un contexto de no juego para conducir el comportamiento de los participantes (mediante la participación, la interacción, la adicción o, incluso, la competición) hacia la consecución de un determinado objetivo de negocio”.

En definitiva, con la gamificación podremos conseguir que nuestros clientes o usuarios realicen aquellos objetivos que hayamos marcado, simplemente por el objetivo de divertirse.

Hoy en día, miles de compañías utilizan la ludificación en sus procesos empresariales, desde sus relaciones con los empleados, hasta con los clientes, pasando por sus comerciales. La ludificación realmente funciona y, por increíble que parezca, es muy efectiva.
No son pocos los ejemplos de una aplicación más que satisfactoria de la misma. En 2011, Volskwagen decidió inventar en China, su mercado más importante, una nueva versión de su ‘people’s car’. Para ello contó con la ayuda de sus clientes, a quienes ofreció una herramienta de diseño y un sistema de puntuaciones. El resultado del mismo: más de 50.000 propuestas diferentes.
Hay otros ejemplos, por ejemplo, en nuestro país, Correos decidió en 2012 rediseñar su web y se le plantearon dos propuestas: contratar a una empresa por miles de euros, o bien, plantear un sistema de gamificación en el que los empleados propusieran nuevos diseños, a cambio de pequeños regalos. El resultado fue un ahorro de un 70\%. Se presentaron más de 50.000 propuestas en un tiempo récord de tiempo.

\subsection{Propuesta.}
Este Trabajo de Fin de Grado estará centrado en el TFG “Estudio de gamificación en una empresa para mejorar la fidelización de los clientes”, elaborado por Dª Cristina Martínez Martínez, graduada durante el curso 2016-2017 en Ingeniería de Organización Industrial por la Universidad de Valladolid.  

El trabajo de esta alumna, centrado en el diseño de un sistema de gamificación aplicado a un posible caso real (restaurante de comida americana), tenía el problema de la imposibilidad de llevarlo a cabo con herramientas profesionales, debido a su coste y a la falta de disponibilidad de las mismas. Por este motivo, el TFG se tuvo que defender contando solo con una serie de prototipos basados en un sistema web. 

Como la gamificación es un área en investigacion y que ofrece muy buenos resultados, la tutora del TFG propuso a la Comisión de Título la realización de este trabajo. Por otro lado, el resultado del mismo puede servir a alumnos de otras facultades para poder utilizar una herramienta de gamificación real.
\subsubsection{Resumen de “Estudio de gamificación en una empresa para mejorar la fidelización de los clientes”}
Se propone la realización de un sistema de gamificación para una cadena de restaurantes de comida americana, que cuenta con un conjunto de franquicias repartidas por todo el territorio español. Es una empresa con muy buenos resultados económicos y cuenta con una amplia y sólida base de clientes, que decide elaborar una aplicación que mejore la opinión de los mismos sobre la marca, ya que han llegado críticas sobre el sistema vigente de cupones y descuentos.

Finalmente, el equipo directivo decide encargar una aplicación basada en la ludificación cuyos objetivos son los siguientes:

\begin{enumerate}
\item Aumentar la fidelización de los clientes.
\item Incentivar las ventas.
\item Mejorar la imagen de la marca en redes sociales.
\item Conseguir nuevos clientes.
\item Mejorar la confianza y satisfacción del cliente. 
\end{enumerate}

Se medirá el cumplimiento de los objetivos anteriores mediante el cálculo de una serie de ratios.\\

\textbf{Público objetivo}\\

El público objetivo estará formado por hombres y mujeres, cuyas edades estén comprendidas entre los doce y los cincuenta y cinco años. 
Los conocimientos tecnológicos esperados en este grupo de edad no tienen por que ser altos, ya que se utilizará una aplicación móvil. Se considerará suficiente por tanto, conocimientos básicos en uso de smartphones y de uso de aplicaciones móviles. En el caso en el que nos encontramos la aplicación a desarrollar será Android.

Los potenciales usuarios de la aplicación la utilizarán para conseguir recompensas y descuentos derivados del uso de la misma.

Finalmente, indicar que se clasificará cada jugador en función de sus gustos y comportamientos, siguiendo la teoría de Battle, haciendo así que la experiencia de usuario sea distinta entre usuarios de distinta categoría. En concreto, se realizarán distintos tipos de juegos.\\

\textbf{Roles}\\

Se proponen dos tipos de roles: Administrador y Cliente.

Las funcionalidades que podrá acceder cada tipo de rol vendrán determinadas en un diagrama de casos de uso que se especificará en el apartado correspondiente. \\

\textbf{Descripción del juego}\\

El juego se centra en un mapa de la Ruta66 americana, aprovechando así la temática del restaurante. Se dividirá el camino en once etapas o niveles, correspondientes a las principales ciudades por las que pasa esta vía, y para avanzar, el cliente deberá completar una serie de actividades, que además otorgarán insignias. Cabe destacar que la dificultad de los niveles irá aumentando gradualmente, con el objetivo de no perder la motivación.

\begin{enumerate}
\item Chicago.
\item Springfield.
\item St. Louis.
\item Tulsa.
\item Oklahoma City.
\item Amarillo.
\item Santa Fe.
\item Albuquerque.
\item Flagstaff.
\item Williams.
\item Los Ángeles.
\end{enumerate}

Todos los niveles excepto el de Chicago tendrán un planteamiento similar:
\begin{itemize}
\item Nivel 1. Chicago: Compuesto por diez casillas. Se activarán dos por cada actividad completada:
	\begin{itemize}
		\item Configuración del perfil del jugador.
		\item Crear un equipo.
		\item Seguir en redes sociales la página del restaurante.
		\item Puntuar la aplicación en la tienda de aplicaciones.
		\item Comentar la aplicación en la tienda de aplicaciones.
	\end{itemize}
\item Resto de niveles: Consistirán en diez misiones:
	\begin{itemize}
		\item Misiones de nivel social: consistirán en tareas como compartir el progreso, subir una foto hecha en el restaurante a las redes sociales, enviar invitaciones…
		\item Misiones de minijuegos: serán dos casillas y consistirá en jugar a un minijuego, dependiente del perfil de jugador.
		\item Misiones de consumo: el jugador deberá realizar consumiciones en los restaurantes de la cadena. Avanzará más o menos casillas en función del gasto que realice.
		\item Misiones de retos especiales: en días señalados, los jugadores podrán participar en retos creados específicamente para ese día.
	\end{itemize}
\end{itemize}

\textbf{Recompensas}\\

Hay tres tipos de recompensa:
\begin{itemize}
\item Puntos: el usuario obtendrá diez puntos por cada casilla que avance. Similarmente, se agrupará la suma total en tres categorías: consumo, social y competitivo.
\item Insignias: cuando el jugador haya completado un nivel completo obtendrá una insignia (matrícula de la ciudad asociada). Habrá, adicionalmente, otras insignias, entre las que se encuentran la familiar o las de grupo.
\item Premios: los premios serán regalos. Se desbloquearán cuando el usuario haya completado un nivel o haya realizado algún reto especial.
\end{itemize}

\textbf{Perfiles de jugador}\\ \cite{iebsctj}

Todos los jugadores tendrán asociado un perfil, que se determinará en el registro de usuario:
\begin{itemize}

\item Triunfador: tendrá como objetivo llevar a cabo las misiones y obtener los premios o recompensas.
\item Explorador: son jugadores que les gusta descubrir o aprender cosas nuevas.
\item Socializadores: más que interés en conseguir los logros, buscan aprovecharlos para entablar relaciones sociales.
\item Killers: buscan ser los primeros en el juego. Quieren destacar sobre otros jugadores.

\end{itemize}

\textbf{Grupos}\\

Los jugadores, similarmente, podrán crear el número que quieran de equipos. Participarán con los mismos a la hora de alcanzar las recompensas.
\chapter{Parte II: Herramientas utilizadas y entorno de trabajo}
\section{Herramientas utilizadas}
\section{Entorno de trabajo}
\section{Comunicación con el servidor}

\chapter{Parte III: Route66App}
\section{Plan de Desarrollo del Software}
\subsection{Metodología de Desarrollo del proyecto}
\subsection{Restricciones}
\subsection{Estructura organizativa}
\subsubsection{Estructura organizativa interna}
\subsubsection{Estructura organizativa externa}
\section{Gestión del proyecto}
\subsection{Planificación del proyecto}
\subsection{Calendario del proyecto}
\subsubsection{Fase de inicio}
\subsubsection{Fase de elaboración}
\subsubsection{Fase de implementación}
\subsection{Relación de artefactos}
\subsection{Control y seguimiento del proyecto.}
\subsection{Gestión de riesgos.}
\subsection{Gestión de configuraciones.}
\subsection{Costes.}

\begin{thebibliography}{a}
\bibitem[IEBSCTiposJugadores]{iebsctj} \textsc{Ferran Altarriba Bertran}, \textit{\url{http://www.iebschool.com/blog/tipos-jugadores-gamification-2-innovacion/}}.  

\bibitem[Old]{old} \textsc{Old, L.},\textit{Confesiones de una oveja bizca.} 1ª ed. Madrid: Naturalistic, 2010. 
\end{thebibliography}

\chapter{Anexos}



\end{document}